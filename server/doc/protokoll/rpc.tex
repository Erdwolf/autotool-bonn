Die Aufrufe sind an Haskell angelehnt in der Form
\begin{alltt}
function_name :: ParameterTyp1 -> ParameterTyp2 -> ResultatTyp
\end{alltt}
dargestellt.

\subsection{\texttt{get\_server\_info}}
\begin{alltt}
get_server_info :: ServerInfo
\end{alltt}

\subsubsection*{Zweck}
Abfrage von grundlegenden Serverinformationen. Dazu geh�rt insbesondere auch die unterst�tzte Protokollversion, so dass Inkompatibilit�ten erkannt werden k�nnen.

\subsubsection*{Parameter}
keine
\subsubsection*{R�ckgabewert}
Der Aufruf liefert Name und Version des Servers sowie die implementierte Version des Protokolls zur�ck. Siehe \texttt{ServerInfo} (\ref{ServerInfo}).

\subsection{\texttt{get\_task\_types}}
\begin{alltt}
get_task_types :: [TaskTree]
\end{alltt}
\subsubsection*{Zweck}
\subsubsection*{Parameter}
keine
\subsubsection*{R�ckgabewert}
Eine Liste von kategorisierten Aufgabentypen. Siehe \texttt{TaskTree} (\ref{TaskTree})

\subsection{\texttt{get\_task\_description}}
\begin{alltt}
get_task_description :: Task -> TaskDescription
\end{alltt}
\subsubsection*{Zweck}
Mit diesem Aufruf wird nach Auswahl des Ausgabentyps n�here Information zu diesem abgefragt.
\subsubsection*{Parameter}
\subsubsection*{R�ckgabewert}

\subsection{\texttt{verify\_task\_config}}
\begin{alltt}
verify_task_config :: Task -> Config
    -> Either Description (Signed (Task, Config))
\end{alltt}
\subsubsection*{Zweck}
\subsubsection*{Parameter}
\subsubsection*{R�ckgabewert}

\subsection{\texttt{get\_task\_instance}}
\begin{alltt}
get_task_instance :: Signed (Task, Config) -> Seed
    -> ((Signed (Task, Instance), Description, Documented Solution)
\end{alltt}
\subsubsection*{Zweck}
\subsubsection*{Parameter}
\subsubsection*{R�ckgabewert}

\subsection{\texttt{grade\_task\_solution}}
\begin{alltt}
grade_task_solution :: Signed (Task, Instance) -> Solution
    -> Either Description (Documented Double)
\end{alltt}
\subsubsection*{Zweck}
\subsubsection*{Parameter}
\subsubsection*{R�ckgabewert}

\endinput
