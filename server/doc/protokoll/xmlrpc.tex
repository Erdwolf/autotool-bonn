%%%%%%%%%%%%%%%%%%%%%%%%%%%%%%%%%%%%%%%%%%%%%%%%%%%%%%%%%%%%%%%%%%%%%%%%%%%%%%
\subsection{XML-RPC}
Die RPC-Aufrufe werden per XML-RPC-Protokoll �bertragen. XML-RPC basiert auf HTTP. Ein RPC-Aufruf ist ein POST-Request an die URL des \emph{autOlat}-Service, bei dem die Parameter und R�ckgabewerte XML-kodiert �bertragen werden. Details zur Kodierung gibt es unter \url{http://www.xmlrpc.com/}.

\subsubsection{Basistypen}
Die folgenden XML-RPC-Basistypen werden verwendet:
\begin{itemize}
\item \texttt{string} -- Zeichenketten -- f�r \texttt{String} und \texttt{Int}
\item \texttt{double} -- Gleitkommazahlen -- f�r \texttt{Double}
\item \texttt{boolean} -- Wahrheitswerte -- f�r \texttt{Bool}
\end{itemize}

\subsubsection{Strukturtypen}
\begin{itemize}
\item \texttt{array} -- Felder -- f�r Listen (\texttt{[$\alpha$]}) und komplexe Datentypen.

Verk�rzte Notation: \texttt{[ $a$, $b$, \ldots\ ]}
\item \texttt{struct} -- records -- f�r komplexe Datentypen

Verk�rzte Notation: \texttt{\{ $k_1$: $v_1$, $k_2$: $v_2$, \ldots\ \}}
\end{itemize}

%%%%%%%%%%%%%%%%%%%%%%%%%%%%%%%%%%%%%%%%%%%%%%%%%%%%%%%%%%%%%%%%%%%%%%%%%%%%%%
\subsection{Abbildung der Datentypen in XML-RPC}
\subsubsection{Basistypen}
%
\begin{tabular}{>{\tt}l>{\tt}l|>{\tt}l>{\tt}l}
Typ    & Beispiel   & XML-RPC-Typ & Beispiel \\
\hline
String & \dq abc\dq & string      & \dq abc\dq \\
Int    & 123        & string      & \dq 123\dq \\
Double & 123.0      & double      & 123.0 \\
Bool   & True       & boolean     & true \\
\relax[$\alpha$] & \relax[\dq a\dq, 42] & array & \relax[\dq a\dq, 42] \\
\end{tabular}\\[1ex]
%
Werte von Typaliasen werden wie die entsprechenden Basistypen kodiert.

\subsubsection{komplexe Datentypen}
Komplexe (algebraische) Datentypen haben einen Namen, Typparameter, falls sie generisch sind, und eine oder mehrere Auspr�gungen. Jede Auspr�gung ist ein Record oder ein Tupel. Die jeweils verwendete Auspr�gung bestimmt die Kodierung der Werte.
\begin{alltt}
data Name \(\alpha\) = Alternative1
            | Alternative2
            \ldots
\end{alltt}

\subsubsection{Records}
Records haben einen Namen und benannte Datenfelder. Beispiel:
\begin{alltt}
data \ldots = Record \{
    field1 :: \(\alpha\),
    fiedl2 :: \(\beta\)
\}
\end{alltt}
Sie werden als \texttt{struct} kodiert, die eine weitere \texttt{struct} enth�lt:
\begin{alltt}
\{
    "Record": \{
         "field1": \ldots,
         "field2": \ldots
    \}
\}
\end{alltt}

\subsubsection{Tupel}
Tupel haben einen Namen und \emph{unbenannte} Datenfelder. Beispiel:
\begin{alltt}
data \ldots = Record \(\alpha\) \(\beta\)
\end{alltt}
Sie werden als \texttt{struct} kodiert, die ein \texttt{array} enth�lt:
\begin{alltt}
\{
    "Record": [
         \ldots,
         \ldots
    ]
\}
\end{alltt}

\subsubsection{Either}
\begin{alltt}
data Either \(\alpha\) \(\beta\) = Left \(\alpha\) | Right \(\beta\)
\end{alltt}
Abweichend von der normalen Kodierung f�r Tupel werden die Werte von \texttt{Either} nicht extra in array verpackt. So wird \texttt{Left \ldots} als
\begin{alltt}
\{
     "Left": \ldots
\}
\end{alltt}
kodiert. Die Kodierung von \texttt{Right \ldots} ist analog.

\endinput
