%%%%%%%%%%%%%%%%%%%%%%%%%%%%%%%%%%%%%%%%%%%%%%%%%%%%%%%%%%%%%%%%%%%%%%%%%%%%%%
\begin{description}
\item[Server] Programm, das als Anbieter von Aufgabensemantik f�r verschiedene Aufgabentypen fungiert. Es nimmt Anfragen vom \emph{Client} entgegen, und beantwortet diese.
\item[Client] Nutzer des \emph{Servers}. Dies kann z.B. eine Lernplattform sein, �ber die Lernenden Zugang zu den vom Server implementierten Aufgaben bekommt.
\item[Aufgabentyp] Ein \emph{Server} kann Semantik f�r verschiedene Aufgabentypen anbieten. Beispiele w�ren das L�sen linearer Gleichungssysteme oder das Faktorisieren einer Zahl. Im Allgemeinen umfasst jeder Aufgabentyp viele konkrete Aufgabenstellungen.
\item[Aufgabenkonfiguration] Zusammen mit dem \emph{Aufgabentyp} definiert die Aufgabenkonfiguration die Rahmenparameter f�r die zugeh�rigen \emph{Aufgabeninstanzen}. Dies k�nnte z.B. die Anzahl der Gleichungen in einem linearen Gleichungssystem sein.
\item[Aufgabeninstanz] Die Aufgabeninstanz ist eine konkrete Aufgabenstellung, also beispielsweise ein lineares Gleichungssystem. Der \emph{Server} kann diese Anhand einer \emph{Aufgabenkonfiguration} und zugeh�rigem \emph{Aufgabentyp} erzeugen. Die Aufgabeninstanz kann dem Endnutzer pr�sentiert und von diesem gel�st werden.
\item[Aufgabenl�sung] Eine Aufgabenl�sung stellt einen L�sungsversuch f�r eine \emph{Aufgabeninstanz} dar. Diese kann vom \emph{Server} auf Korrektheit gepr�ft werden.
\item[Bewertung] Der \emph{Server} stellt fest, ob die L�sung korrekt war oder nicht.
\item[Gr��e] Neben der eigentlichen \emph{Bewertung} gibt es die M�glichkeit, Aufgabenl�sungen eine zus�tzliche Zahl, die \emph{Gr��e} zuzuordnen, beispielsweise die Anzahl der L�sungsschritte einer L�sung. Diese ist von sekund�rer Bedeutung - eine korrekte L�sung mit 100 Schritten ist genauso richtig wie eine mit 20 Schritten. Man kann die Zahl aber f�r Bestenlisten f�r die jeweilige Aufgabe verwenden.
\end{description}
\endinput
