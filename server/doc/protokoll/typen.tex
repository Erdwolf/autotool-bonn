%%%%%%%%%%%%%%%%%%%%%%%%%%%%%%%%%%%%%%%%%%%%%%%%%%%%%%%%%%%%%%%%%%%%%%%%%%%%%%
\subsection{Basistypen}
\begin{tabular}{>{\tt}l|l}
Typ        & Bedeutung\\
\hline
String     & UTF-8-kodierte Zeichenkette\\
Int        & vorzeichenbehafteter 32-bit Ganzzahlwert\\
Double     & 64-bit Gleitkommazahl\\
\relax[$\alpha$] & Liste von Elementen vom Typ $\alpha$\\
($\alpha$, $\beta$) & Paar von Werten vom Typ $\alpha$ und $\beta$\\
($\alpha$, $\beta$, $\gamma$) & Tripel\\
\end{tabular}

%%%%%%%%%%%%%%%%%%%%%%%%%%%%%%%%%%%%%%%%%%%%%%%%%%%%%%%%%%%%%%%%%%%%%%%%%%%%%%
\subsection{Aliase}
\begin{tabular}{>{\tt}l@{ = }>{\tt}l|l}
Alias       & Typ    & Bedeutung\\
\hline
Name        & String & uninterpretierter Name\\
Task        & String & Name eines Aufgabentyps\\
Seed        & String & vom Nutzer abh�ngiger String zur Generierung von Aufgaben\\
Signature   & String & vom Server erzeugte Signatur\\
Config      & String & Aufgabenkonfiguration\\
Description & String & XML-Dokument mit textueller Beschreibung\\
Solution    & String & Aufgabenl�sung\\
\end{tabular}

%%%%%%%%%%%%%%%%%%%%%%%%%%%%%%%%%%%%%%%%%%%%%%%%%%%%%%%%%%%%%%%%%%%%%%%%%%%%%%
\subsection{\texttt{Documented $\alpha$}}\label{Documented}
\begin{alltt}
data Documented \(\alpha\) = Documented \{
    contents :: \(\alpha\),
    documentation :: Description
\}
\end{alltt}

\subsubsection*{Beschreibung}
Dieser Typ dient der Annotation eines Wertes mit einer textuellen Beschreibung f�r den Nutzer.

%%%%%%%%%%%%%%%%%%%%%%%%%%%%%%%%%%%%%%%%%%%%%%%%%%%%%%%%%%%%%%%%%%%%%%%%%%%%%%
\subsection{\texttt{Signed $\alpha$}}\label{Signed}
\begin{alltt}
data Signed \(\alpha\) = Signed \{
    contents :: \(\alpha\),
    signature :: Signature
\}
\end{alltt}

\subsubsection*{Beschreibung}
Dieser Typ dient der Annotation eines Wertes mit einer Signatur. Die Signatur wird vom Server gepr�ft, wenn ein vom Server gelieferter Wert vom Client an den Server zur�ckgegeben wird. F�r den Client ist die Signatur ohne Bedeutung; er muss sie allerdings aufbewahren.

%%%%%%%%%%%%%%%%%%%%%%%%%%%%%%%%%%%%%%%%%%%%%%%%%%%%%%%%%%%%%%%%%%%%%%%%%%%%%%
\subsection{\texttt{Instance}}\label{Instance}
\begin{alltt}
data Instance = Instance \{
    tag :: String,
    contents :: String
\}
\end{alltt}

\subsubsection*{Beschreibung}
Dieser Typ beschreibt zusammen mit einem Aufgabentyp eine Aufgabeninstanz. Wie die Datenfelder genau verwendet werden, obliegt dem Server.

%%%%%%%%%%%%%%%%%%%%%%%%%%%%%%%%%%%%%%%%%%%%%%%%%%%%%%%%%%%%%%%%%%%%%%%%%%%%%%
\subsection{\texttt{ServerInfo}}\label{ServerInfo}
\begin{alltt}
data ServerInfo = ServerInfo \{
    protocol_version :: Version,
    server_name :: Name,
    server_version :: Version
\}
\end{alltt}
\subsubsection*{Beschreibung}
Werte dieses Typs enthalten den Servernamen und dessen Version, sowie die Protokollversion, die der Server unterst�tzt. Siehe \texttt{Version} (\ref{Version}).

%%%%%%%%%%%%%%%%%%%%%%%%%%%%%%%%%%%%%%%%%%%%%%%%%%%%%%%%%%%%%%%%%%%%%%%%%%%%%%
\subsection{\texttt{TaskTree}}\label{TaskTree}
\begin{alltt}
data TaskTree = Task \{
    task_name :: Task
\}
| Category \{
    category_name :: Name,
    sub_trees :: [TaskTree]
\}
\end{alltt}

\subsubsection*{Beschreibung}
Ein Wert dieses Typs enth�lt entweder den Namen eines Aufgabentyps, oder beschreibt eine ganze Unterkategorie von Aufgaben. In diesem Fall enth�lt er den Namen der Kategorie sowie eine Liste von Teilaufgaben und weiteren Unterkategorien.

%%%%%%%%%%%%%%%%%%%%%%%%%%%%%%%%%%%%%%%%%%%%%%%%%%%%%%%%%%%%%%%%%%%%%%%%%%%%%%
\subsection{\texttt{Version}}\label{Version}
\begin{alltt}
data Version = Version \{
    major :: Int,
    minor :: Int,
    micro :: Int
\}
\end{alltt}
\subsubsection*{Beschreibung}
Dieser Typ beschreibt eine Versionsnummer. Versionsnummern bestehen aus drei Komponenten. Dabei gilt folgende Konvention:

\begin{itemize}
\item Bei inkompatiblen �nderungen soll die major-Version erh�ht werden.
\item Bei abw�rtskompatiblen �nderungen soll die minor-Version erh�ht werden.
\item Bei kompatiblen �nderungen soll die micro-Version erh�ht werden.
\end{itemize}

%%%%%%%%%%%%%%%%%%%%%%%%%%%%%%%%%%%%%%%%%%%%%%%%%%%%%%%%%%%%%%%%%%%%%%%%%%%%%%
\subsection{\texttt{TaskDescription}}\label{TaskDescription}
\begin{alltt}
data TaskDescription = TaskDescription {
    task_sample_config :: Documented Config,
    task_scoring_order :: ScoringOrder
}
\end{alltt}
\subsubsection*{Beschreibung}
Werte dieses Typs beschreiben einen Aufgabentyp n�her. Dabei wird eine Beispielkonfiguration zur�ckgeliefert, die mindestens syntaktisch korrekt sein soll, dabei nach M�glichkeit auch eine sinnvolle Aufgabenstellung ergibt.

Daneben wird angegeben, in welcher Reihenfolge Punkte f�r Toplisten ausgewertet werden sollen. Siehe \texttt{ScoringOrder} (\ref{ScoringOrder}).

%%%%%%%%%%%%%%%%%%%%%%%%%%%%%%%%%%%%%%%%%%%%%%%%%%%%%%%%%%%%%%%%%%%%%%%%%%%%%%
\subsection{\texttt{ScoringOrder}}\label{ScoringOrder}
\begin{alltt}
data ScoringOrder = Increasing
                  | None
                  | Decreasing
\end{alltt}
\subsubsection*{Beschreibung}
Sortierreihenfolge f�r Bestenlisten.
\begin{itemize}
\item \texttt{Decreasing} -- absteigende Sortierung. Das hei�t, eine gr��ere Gr��e ist besser als eine kleinere.
\item \texttt{Increasing} -- aufsteigende Sortierung. Kleinere Gr��en sind besser.
\item \texttt{None} -- keine Angabe, z.B. weil die Sortierreihenfolge von der Aufgabenstellung abh�ngt, oder weil die berechnete Gr��e nicht variiert.
\end{itemize}

%%%%%%%%%%%%%%%%%%%%%%%%%%%%%%%%%%%%%%%%%%%%%%%%%%%%%%%%%%%%%%%%%%%%%%%%%%%%%%
\subsection{\texttt{Either $\alpha$ $\beta$}}\label{Either}
\begin{alltt}
data Either \(\alpha\) \(\beta\) = Left \(\alpha\)
                | Right \(\beta\)
\end{alltt}
\subsubsection*{Beschreibung}
\texttt{Either $\alpha$ $\beta$} stellt eine Alternative zwischen einem Wert vom Typ $\alpha$ oder einem Wert vom Typ $\beta$ dar.

\endinput
