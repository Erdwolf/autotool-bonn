\documentclass{llncs}

\usepackage{amsmath}

\renewcommand{\baselinestretch}{1.1}
\parskip 1mm
\parindent 0mm
\textheight 25cm \voffset -1in
\textwidth 16cm \hoffset -.5in

\newcommand{\PCP}{\operatorname{PCP}}
\newcommand{\from}{\leftarrow}

\author{Mario Schmidt \inst{1}
  \and 
  Johannes Waldmann \inst{2}
}
\institute{
  Universit\"at Leipzig
  \and
  HTWK Leipzig
}
\title{Post's Correspondence Problem \\
  for Instances $[(w,0),(0,1),(1,w)]$
}

\begin{document}
\maketitle

\begin{abstract}
  We determine the solvability of Post's Correspondence Problem
  for instances of the form  $[(w,0),(0,1),(1,w)]$ 
  with $w \in\Sigma^*$ for $\Sigma= \{0,1\}$.
  These instances are unsolvable 
  if and only if $w \in 0 \Sigma^* 0 \cup 1 \Sigma^* 1$.
\end{abstract}

\section{Introduction}

$\PCP(2)$ decidable, $\PCP(7)$ undecidable, what about $\PCP(3)$?
How difficult can it get? Look for \emph{busy beavers} ---
one of them is $[(001,0),(0,1),(1,001)]$, found by brute force enumeration.
Therefore, investigate instances  $[(w,0),(0,1),(1,w)]$.
Indeed, some have long shortest solutions (TODO: Bespiele).

\section{Notation and Preliminaries}

Word, morphism. 

$\PCP$ \emph{instance} is a pair $(g,h)$ of morphisms $\Gamma\to\Sigma$.
A \emph{solution} of  $(g,h)$ is $w\in\Gamma^+$ such that $g(w)=h(w)$.

$\PCP(k)$ is set of instances with $|\Gamma|=k$.

\section{One-Sided Differences}

Restrict to $w = 0x1$. Then any solution starts with $A$ and ends with $C$,
and cannot \"uberholen:

For instances  $[(w,0),(0,1),(1,w)]$,
and any $u\in\Gamma^*$, we have $|g(u)| - |h(u)| \equiv 0 \pmod |w|-1$.

Instance $I=(g,h)$ defines relation $\to_I$ on $\Sigma^*$ by
\[
   u \to_I v \iff \exists x\in\Gamma: u g(x) = h(x) v.
\]

Then $I$ solvable iff $\epsilon \to_I^+ \epsilon$.


\section{Extended Alphabet}

Set $\Sigma_2 = \{0,1,2\} = \Sigma\cup\{2\}$.

For given instance $I=[(w,0),(0,1),(1,w)]$, define new instance
\[
  I_2 = [(w,0),(0,1),(1,w),(2,0),(1,2)].
\]
To each solution of $I_2$ there corresponds a solution of $I$.


\section{Separated Solutions}

Looking for solutions of $I_2$ of the form
\[
  \{ (w,0),(0,1),(2,0),(1,2) \}^*  \{ (1, w),(0,1),(2,0),(1,2) \}^*,
\]
that is, all applications of $(w,0)$ are to the left
of all applications of $(1,w)$.


\section{Cycling}

Note that $[(0,1),(2,0),(1,2)]$ is a cycle of length three.
More efficient notation by doing all computations ``modulo 3''.

Denote by $(\cdot)_+$ 
the alphabetic morphism $0 \mapsto 1, 1 \mapsto 2, 2 \mapsto 0$,
and by $(\cdot)_-$ its inverse.

Let $\sim$ be \emph{cyclic conjugacy}:
the reflexive and transitive closure of the relation
$\{ (xy, y_+x) \mid x,y\in\Sigma_2^* \}$.

Claim. $u \sim v$ iff $u \to_I'^* v$ for $I'=[(0,1),(2,0),(1,2)]$.

\section{Rewriting}

For fixed $w$ (from the instance),
define rewriting system $R = \{ 0 \to w_+, 1 \to w_-, 2 \to w \}$.

Claim. $u \to_R v$ implies $u \to_{I_2} v$.

(note: $\to_R$ is the rewriting relation,
$\to_{I_2}$ is the ``rotating differences'' relation.)

Proof. Example: let $u \to_R v$ by $u = p 1 q, p w_- q = v$.
Then $u = p 1 q \to_I^* 1 q p_+ \to_I q p_+ 2 \to_I^* 2 q_+ p
\to_I q_+ p 0 \to_i^* 0 q p_+ \to_I q p_+ w \to_I^* p_+ w q_+
\to_I^* p_- w_+ q_- \to_I^* p w_- q = v$.

\section{Types of Separated Solutions}

Let $w = 0x1$. Looking for $0x \to_R^* u \sim v \from_R^* x1$.

In fact, distinguish a few cases:
$x = \epsilon, x=0, x=1, x=0y0, x=0y1, x=1y0, x=1y1$.

Example $w=0x1=00y11$:

\[
\begin{array}{cllllll}
        & \fbox{0} 0 y   1 \\
  \to_R & \fbox{1} 1 y_+ 2 & 2 0 y   1 \\
  \to_R & 2 \fbox{2} y_- 0 & 0 1 y_+ 2 & 2 0 y   1 \\
  \to_R & 2 0 0 y & 1 \fbox{1} y_- 0   & 0 1 y_+ 2 & 2 0 y   1 \\
  \to_R & 2 0 0 y & 1 2 2 y_- & 0 \fbox{0} y_- 0 & 0 1 y_+ 2 & 2 0 y   1 \\
  \to_R & 2 0 0 y & 1 2 2 y_- & 0 1 1 y_+ & 2 2 y_- 0 & 0 1 y_+ 2 & 2 0 y   1 \\
  \sim      & 0 y 1 2 & 2 y_- 0 1 & 1 y_+  2 2 & y_- 0  0 1 & y_+ 2 2 0 & y   1 1 2 \\
  \from_R   & 0 y 1 2 & 2 y_- 0 1 & 1 y_+  \fbox{1} 1 & y_+ 2 2 0 & y   1 1 2 \\
  \from_R   & 0 y 1 2 & 2 y_- 0 1 & 1 y_+  \fbox{0} 0 & y   1 1 2 \\
  \from_R   & 0 y 1 2 & 2 y_- 0 1 & 1 y_+  \fbox{2} 2 \\
  \from_R   & 0 y 1 2 & 2 y_- 0 \fbox{0} \\
  \from_R   & 0 y 1 \fbox{1}
\end{array}
\]


\section{Conclusion and Open Problems}

Only linear complexity for $[(w,0),(0,1),(1,w)]$.
Confirms the conjecture that $\PCP(3)$ complexity is at most quadratic.

$\PCP(3)$ as a whole remains open.
Perhaps treat the case $[(u,0),(0,1),(1,v)]$ next.
Can adapt the ``morphistic'' approach.

\end{document}

