\begin{slide}{Erweiterungen (I)}
\begin{itemize}
\item
  Bewertung der Einsendung nicht nur qualitativ (boolean),
  sondern zus�tzlich quantitativ (f�r Highscore)
\item
  Dokumentation der Ausgabe nicht als String,
  sondern strukturiert (z. B. HTML)
\item
  Eingabe nicht String, sondern strukturiert (XML):
  Benutzung schemagesteuerter Editoren.
\end{itemize}

\end{slide}

\begin{slide}{Erweiterungen (II)}
  Eingabe und Bewertung nicht als Ganzes,
  sondern schrittweise.  Viele Aufgaben gestatten den Begriff
  \emph{partiell korrekte L�sung}
  ($=$ kann zu total korrekter fortgesetzt werden). Bsp:
  \begin{itemize}\itemsep 2pt
  \item
    COL: konfliktfreie F�rbung f�r Teilgraphen
  \item
    PCP: L�sungswort, das Pr�fixbedingung erf�llt.
  \end{itemize}
Vorteile:
\begin{itemize}\itemsep 2pt
\item
  Eingabe einfacher Schritte: point/click (statt Text)
\item
  Feedback nach jedem L�sungsschritt
\item
  bessere Behandlung mehrerer Quantorenwechsel
  (z. B. Pumping-Lemma) (Student: $\exists$, autotool: $\forall$)
\end{itemize}


\end{slide}
